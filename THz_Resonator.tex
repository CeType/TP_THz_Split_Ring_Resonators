\documentclass[aps,prb,twocolumn,groupedaddress]{revtex4-2}
%superscriptaddress or groupedaddress

\begin{document}

%Title of paper
\title{THz spectroscopy of metallic split ring resonnator}

%\author section
\author{Quentin Bravo\thanks{This work was supported by the XYZ Foundation.}}
\email{quentin.bravo@etu.chimieparistech.psl.eu}
\affiliation{Laboratoire de Physique, École Normale Supérieure, 
CNRS, Université PSL, Sorbonne Université, Université de Paris, 75005 Paris, France}

\author{Matéo Rivera}
\email{mateo.rivera@polytechnique.edu}
\affiliation{Laboratoire de Physique, École Normale Supérieure, 
CNRS, Université PSL, Sorbonne Université, Université de Paris, 75005 Paris, France}

\author{Juliette Mangeney}
\email{juliette.mangeney@phys.ens.fr}
\affiliation{Laboratoire de Physique, École Normale Supérieure, 
CNRS, Université PSL, Sorbonne Université, Université de Paris, 75005 Paris, France}


\date{\today}

%Abstract session
\begin{abstract}
% insert abstract here

\end{abstract}

% insert suggested keywords - APS authors don't need to do this
%\keywords{}

%\maketitle must follow title, authors, abstract, and keywords
\maketitle

% body of paper here - Use proper section commands
% References should be done using the \cite, \ref, and \label commands
\section{Introduction}
\cite{PrinciplesTerahertzScience2009}
\cite{neuTutorialIntroductionTerahertz2018}
\cite{lindenPhotonicMetamaterialsMagnetism2006a}
\\
\section{Experimental}


% Put \label in argument of \section for cross-referencing
%\section{\label{}}
%\subsection{}
%\subsubsection{}

% If in two-column mode, this environment will change to single-column
% format so that long equations can be displayed. Use
% sparingly.
%\begin{widetext}
% put long equation here
%\end{widetext}

% figures should be put into the text as floats.
% Use the graphics or graphicx packages (distributed with LaTeX2e)
% and the \includegraphics macro defined in those packages.
% See the LaTeX Graphics Companion by Michel Goosens, Sebastian Rahtz,
% and Frank Mittelbach for instance.
%
% Here is an example of the general form of a figure:
% Fill in the caption in the braces of the \caption{} command. Put the label
% that you will use with \ref{} command in the braces of the \label{} command.
% Use the figure* environment if the figure should span across the
% entire page. There is no need to do explicit centering.

% \begin{figure}
% \includegraphics{}%
% \caption{\label{}}
% \end{figure}

% Surround figure environment with turnpage environment for landscape
% figure
% \begin{turnpage}
% \begin{figure}
% \includegraphics{}%
% \caption{\label{}}
% \end{figure}
% \end{turnpage}

% tables should appear as floats within the text
%
% Here is an example of the general form of a table:
% Fill in the caption in the braces of the \caption{} command. Put the label
% that you will use with \ref{} command in the braces of the \label{} command.
% Insert the column specifiers (l, r, c, d, etc.) in the empty braces of the
% \begin{tabular}{} command.
% The ruledtabular enviroment adds doubled rules to table and sets a
% reasonable default table settings.
% Use the table* environment to get a full-width table in two-column
% Add \usepackage{longtable} and the longtable (or longtable*}
% environment for nicely formatted long tables. Or use the the [H]
% placement option to break a long table (with less control than 
% in longtable).
% \begin{table}%[H] add [H] placement to break table across pages
% \caption{\label{}}
% \begin{ruledtabular}
% \begin{tabular}{}
% Lines of table here ending with \\
% \end{tabular}
% \end{ruledtabular}
% \end{table}

% Surround table environment with turnpage environment for landscape
% table
% \begin{turnpage}
% \begin{table}
% \caption{\label{}}
% \begin{ruledtabular}
% \begin{tabular}{}
% \end{tabular}
% \end{ruledtabular}
% \end{table}
% \end{turnpage}

% Specify following sections are appendices. Use \appendix* if there
% only one appendix.
%\appendix
%\section{}

% If you have acknowledgments, this puts in the proper section head.
%\begin{acknowledgments}
% put your acknowledgments here.
%\end{acknowledgments}

% Create the reference section using BibTeX:
\bibliography{THz_Resonator}

\end{document}
%
% ****** End of file apstemplate.tex ******

